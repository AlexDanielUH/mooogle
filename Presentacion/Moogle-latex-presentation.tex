\documentclass{beamer}
% Configuración de la presentación
\usetheme{Hannover}
\usecolortheme{dolphin}
\usepackage{graphicx}
\usepackage{eso-pic}
% Paquetes necesarios
\usepackage{graphicx}
\usepackage{amsmath}
\usepackage{url}
\usepackage{tikz}
\usetikzlibrary{fadings}
\usepackage{tikz}
\usetikzlibrary{shapes.geometric, arrows}
\usetikzlibrary{calc}

% Título de la presentación
\title{Moogle}
\subtitle{...Encuentra respuestas en un abrir y cerrar de ojos...}

% Autor y afiliación
\author{Alex Daniel Arbolaez}
\institute{Universidad de la Habana}

% Fecha de la presentación
\date{2023}

\begin{document}
\addtobeamertemplate{background canvas}{}{%
    \begin{tikzpicture}[remember picture,overlay]
        \node[opacity=0.5, at=(current page.south east), anchor=south east] {
            \includegraphics[width=2cm,height=2cm,keepaspectratio]{matcom.png} % Reemplaza 'matcom.png' con la ubicación del archivo de imagen del logo
        };
    \end{tikzpicture}
}
% Página de título
\begin{frame}
  \titlepage
\end{frame}


\begin{frame}
\frametitle{Moogle}

\begin{center}
\begin{tikzpicture}

\node (moogle) at (0,0) {\includegraphics[width=5cm]{Moogle.jpg}};
\draw [ultra thick, white] ($(moogle.south west) + (0.2,0.2)$) rectangle ($(moogle.north east) - (0.2,0.2)$);
\node (caption) at (0,-3) {\large Una imagen del buscador Moogle};
%\draw [->, line width=0.5mm, blue] ($(caption.north) + (0,0.2)$) -- ($(moogle.south) + (0,-0.2)$);

\end{tikzpicture}
\end{center}	

\end{frame}

% Índice
\begin{frame}{Índice}
  \tableofcontents
\end{frame}

% Sección 1: Proceso de búsqueda
\section{Qu\'e es Moogle}

\begin{frame}{Qu\'e es Moogle}
  \begin{enumerate}
    \item Es un buscador de documentos del formato *.txt, que dada una
busqueda del usuario, devuelve los resultados mas relevantes de dicha
busqueda en una base de datos (carpeta).
  
  \end{enumerate}
\end{frame}

% Sección 2: Herramientas utilizadas
\section{Herramientas utilizadas}

\begin{frame}{Herramientas utilizadas}
  \begin{itemize}
    \item \textbf{Clase Archivo}:Se utiliza para representar documentos en un motor de b\'usqueda.
    \item \textbf{Clase Moogle}: \'Esta clase es el n\'ucleo de la funcionalidad del motor de b\'usqueda y 
se encarga de coordinar los diferentes componentes del programa. El
m\'etodo \texttt{"}Query\texttt{"} se encarga de realizar la b\'usqueda.
    \item \textbf{Distancia de Levenshtein}: medida de distancia entre dos cadenas de caracteres.
    \item \textbf{Clase Tools}: Esta contiene una serie de funciones las cuales son necesarias para
lograr un mejor ecosistema del proyecto y poder actualizar y mejorar sin da\~nar la infraestructura del proyecto.
  \end{itemize}
\end{frame}

% Sección 3: Proceso de creación del buscador
\section{¿Qu\'e es el modelo vectorial?}

\begin{frame}{¿Qu\'e es el modelo vectorial?}
  \begin{itemize}
    \item El modelo
vectorial es una t\'ecnica de recuperaci\'on de informaci\'on que representa documentos y consultas como vectores en un espacio vectorial,
donde cada dimensi\'on corresponde a un t\'ermino del vocabulario del
corpus. Esta t\'ecnica es \'util porque permite medir la similitud entre
vectores para determinar la relevancia de los documentos para una
consulta dada. La combinaci\'on de TF y IDF se conoce
como TF-IDF. TF-IDF es una medida de la importancia relativa de
un t\'ermino en un documento o en una consulta en el contexto de
un corpus de documentos.
  \end{itemize}
\end{frame}

% Página de agradecimientos
\begin{frame}{¡Gracias por su atención!}
  \begin{center}
    \Large FIN
  \end{center}
\end{frame}

\end{document}  
