\documentclass[12]{article}
\usepackage[utf8]{inputenc}
\usepackage{graphicx}
\usepackage{xcolor}
\usepackage[margin=25mm,top=30mm,bottom=30mm]{geometry}

\usepackage{fancyhdr}


% Configuración del encabezado y pie de página
\pagestyle{fancy}
\fancyhf{}
\rhead{Alex Daniel Arbolaez}
\lfoot{Moogle}
\title{Moogle}
\author{Alex Daniel Arbolaez Sabater}
\date{La Hbana,Cuba}

\begin{document}

\begin{titlepage}
\centering
\includegraphics[width=1 \textwidth]{Moogle.png} 
\vspace{2cm}

{\Huge \textbf{Moogle} \par}
\vspace{0.5cm}
{\Large ...Encuentra respuestas en un abrir y cerrar de ojos...\par}
\vspace{1.5cm}

Autor: Alex D. Arbolaez Sabater
Universidad: Universidad de La Habana
La Habana, Cuba
\\
\includegraphics[width=0.3 \textwidth]{havmun01.png} 

\vfill

\end{titlepage}

\section{}\LARGE \textcolor{blue}{\textsc{"La programación es una carrera en la que uno nunca deja de aprender." - Richard Branson}}

\newpage
\tableofcontents
\newpage
\section{¿Qué es Moogle?}
\subsection{Introducción}
Es un buscador de documentos del formato *.txt, que dada una busqueda del usuario, 
devuelve los resultados mas relevantes de dicha busqueda en una base de datos (carpeta).
\\
\\

\subsection{¿Cómo se ejecuta?}
Para poder ejecutar el proyecto:
\\
1- Abrir la carpeta donde se encuentra el proyecto
\\
2- Abrir Terminal / Consola y escribir el siguiente comando:
\\
- La terminal de Linux:
\\
  bash
\\make dev
\\ 
\\
-Si estás en Windows
\\ 
bash
\\
dotnet watch run --project MoogleServer
\\
\\
3 - Abrir en el navegador la direccion que ofrece

\newpage
\section{¿Cómo funciona?}
\subsection{Clase Archivo}

\{
Se utiliza para representar documentos en un motor de búsqueda. Mediante procesos de 
calculos logramos extraer informacion de la base de datos que nos seran utiles y esa 
informacion la devolvemos en un metodo que se llama TF\_IDF que contiene la informacion 
escencial para realizar la busqueda.Para hacer ésto comienza representando cada documento 
como un vector que almacena solo la informacion necesaria para realizar las búsquedas. Los 
documentos son normalizados eliminando caracteres complejos de entender por el 
compilador, los espacios y las tildes, obteniendo solamente la lista de palabras de cada
documento. Luego se realiza el cálculo del TF(Termine Frequence) y del IDF(Inverse Document 
Frequence).
\}

\subsection{Clase Moogle}
\{
Ésta clase es el núcleo de la funcionalidad del motor de búsqueda y se encarga de coordinar los 
diferentes componentes del programa. El método "Query" se encarga de realizar la búsqueda. 
Toma una cadena de consulta como parámetro y devuelve un objeto SearchResult que 
contiene una lista de elementos de búsqueda y una sugerencia de búsqueda. La búsqueda se 
realiza utilizando los métodos de las otras clases, incluyendo la normalización de la cadena de 
consulta y la frecuencia de las palabras, la identificación y el trabajo con los operadores, y el 
cálculo del puntaje de relevancia de cada documento en función de la consulta. Finalmente, se 
seleccionan los elementos de búsqueda más relevantes y se devuelven en el objeto 
SearchResult.Al final, los resultados de la búsqueda se ordenan por puntaje de relevancia y se 
muestran los 10 primeros resultados (o menos si hay menos de 10 resultados). Mientras 
compila el programa este se encarga de procesar todas las herramientas necesarias para el 
correcto funcionamiento de nuestro buscador.
\}
\subsection{Clase Tools}
\{
Esta contiene una serie de funciones las cuales son necesarias para lograr un mejor ecosistema 
del proyecto y poder actualizar y mejorar sin dañar la infraestructura del proyecto.
Funciones como:
-ExtraePlabras
-PalabrasSinRepetir
-Direccion 
\}

\newpage
\section{¿Qué es el modelo vectorial?}
El modelo vectorial es una técnica utilizada en recuperación de información que se basa en la 
representación de los documentos y las consultas como vectores en un espacio vectorial. En 
este espacio, cada dimensión representa un término del vocabulario del corpus (conjunto de 
documentos) y el valor de cada componente del vector indica la importancia del término en el 
documento o consulta. Para construir la representación vectorial de un documento, se utiliza 
una técnica de ponderación de términos, como la frecuencia de término inversa (TF-IDF), que 
tiene en cuenta la frecuencia de los términos en el documento y en el corpus para asignar un 
peso a cada término. De esta forma, los términos más importantes para un documento tienen 
un mayor peso en su representación vectorial. Cuando se recibe una consulta de búsqueda, se 
representa también como un vector en el espacio vectorial utilizando la misma técnica de 
ponderación de términos. Entonces, se puede medir la similitud entre el vector de consulta y 
los vectores de los documentos utilizando una medida de similitud, como la similitud del 
coseno. La similitud del coseno mide el ángulo entre dos vectores y proporciona una medida 
de la similitud entre ellos. Cuanto más cercanos sean los vectores, mayor será la similitud del 
coseno y mayor será la probabilidad de que el documento sea relevante para la consulta.
En resumen, el modelo vectorial es una técnica de recuperación de información que 
representa documentos y consultas como vectores en un espacio vectorial, donde cada 
dimensión corresponde a un término del vocabulario del corpus. Esta técnica es útil porque 
permite medir la similitud entre vectores para determinar la relevancia de los documentos 
para una consulta dada.
La idea detrás de IDF es que los términos raros son más importantes para la comprensión del 
contenido de un documento que los términos comunes.
La combinación de TF y IDF se conoce como TF-IDF. TF-IDF es una medida de la importancia 
relativa de un término en un documento o en una consulta en el contexto de un corpus de 
documentos. Se calcula multiplicando la frecuencia de término (TF) por la frecuencia inversa 
de documento (IDF):
En nuestro motor de búsqueda lo utilizamos para medir la relevancia de los documentos en 
función de las consultas de los usuarios. Se calcula el valor de TF-IDF para todos los términos 
en la consulta y en cada documento en la colección, y se devuelve una lista de documentos 
ordenados por su similitud con la consulta.
\newpage
\section{El objetivo de Moogle\!} Es realizar búsquedas en el interior de varios archivos .txt y en función 
del contenido de los mismos, mostrar los resultados más relevantes de acuerdo a la búsqueda 
que usted haya realizado. Para esto, usted debe copiar los archivos .txt a los cuales quiera 
realizarle la búsqueda en la carpeta Content que aparece en la raiz del proyecto. La cantidad 
mínima de archivos .txt que el proyecto debe tener en la carpeta Content para funcionar de 
manera correcta es de 2 archivos. Los cuales ya se encuentran en dicha carpeta. (Siéntase libre 
de borrarlos y copiar sus propios archivos .txt, el código está preparado para trabajar con 
cualquier archivo .txt que usted provea, mientras la cantidad mínima de estos sean 2). En 
cuanto a la cantidad máxima no debería tener ningún problema.
Mi codigo se utiliza mucho el diccionario en C\# pues es una estructura de datos que permite 
almacenar elementos en pares clave-valor. La clave es un valor único que se utiliza para 
identificar el elemento, mientras que el valor es el elemento en sí mismo. 
Hay varias razones por las que podrías querer usar un diccionario en lugar de otro tipo de 
variable para almacenar elementos:
1. Búsqueda eficiente: Los diccionarios en C\# están diseñados para permitir una búsqueda 
eficiente de elementos por clave. Esto significa que puedes buscar un elemento en el 
diccionario en tiempo constante, independientemente del tamaño del diccionario.
3. Fácil acceso a los elementos: Los diccionarios en C\# proporcionan una forma fácil de acceder 
a los elementos almacenados por clave. Esto significa que puedes acceder a cualquier 
elemento en el diccionario simplemente proporcionando su clave.
En resumen, los diccionarios en C\# son una estructura de datos muy útil para almacenar 
elementos en pares clave-valor. Son eficientes en la búsqueda, flexibles en el tipo de 
elementos que pueden almacenar, y proporcionan un fácil acceso y actualización de los 
elementos almacenados.
Este documento se deja abierto a próximas actualizaciones pero por ahora este es el resumen 
del funcionamiento y ecosistema del código…..


\end{document}
